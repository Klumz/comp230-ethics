\documentclass{scrartcl}

\usepackage[hidelinks]{hyperref}
\usepackage[none]{hyphenat}

\title{Essay Proposal}
\subtitle{COMP230-Ethics and Professionalism}

\author{Michail Karakasis}

\begin{document}

\maketitle

\section*{When is it morally justified to retrospectively replace offensive symbolism in mainstream AAA games?}

While the game industry grows in numbers, so does its influence. There have been multiple cases of controversial symbolism in video games in the past (e.g. Wolfenstein 3D), however has become more prominent during these recent years. A brief example of this would be from the first-person shooter game Destiny 2, where a pair of gauntlets in-game were patched for allegedly resembling a hate symbol from a satiric "religion" known as Kek. Are such actions necessary? Do they contribute to the future betterment of the video games industry and its wide audience? Such topics will be tackled in this essay.
% Add details as appropriate.

\section*{Paper 1}
% This is an example! Replace the details with a paper relevant to your chosen topic.
\begin{description}
\item[Title:]Signs, Symbols, Games, and Play
\item[Citation:] \cite{davidmyers}
\item[Abstract:] This article justifies the study of video games with reference to the importance of the study of representations and the study of play.
\item[Web link:]\url{https://www.researchgate.net/publication/247780981_Signs_Symbols_Games_and_Play}
\item[Full text link:] \url{http://luisfilipeteixeira.com/fileManager/file/rev%20Games%20and%20Culture_signs%20symbols%20games%20and%20play_Myers.pdf}
\item[Comments:] 
\end{description}

\section*{Paper 2}
\begin{description}
\item[Title:]Symbolic representation of game world state: Toward real-time planning in games
\item[Citation:] \cite{jefforkin}
\item[Abstract:] As the game development effort scales, AI developers are facing new challenges in terms of implementation, workflow, and game design. The needs of today’s games are outgrowing the typical techniques of modeling behavior with Finite State Machines and Rule Based Systems. This paper argues that a regressive real-time planning system is better suited to address the challenges game developers are facing, and presents symbolic representation strategies that
we have employed to allow planning in practice in today’s games. 
\item[Web link:]\url{https://www.cs.auckland.ac.nz/courses/compsci767s2c/resources/Papers/WS04-04-006.pdf}
\item[Full text link:]\url {https://www.cs.auckland.ac.nz/courses/compsci767s2c/resources/Papers/WS04-04-006.pdf}
\item[Comments:]
\end{description}

\section*{Paper 3}
\begin{description}
\item[Title:]Understanding Video Games: The Essential Introduction
\item[Citation:] \cite{simon}
\item[Abstract:] From Pong to PlayStation 3 and beyond, Understanding Video Games is the first general introduction to the exciting new field of video game studies. This textbook traces the history of video games, introduces the major theories used to analyze games such as ludology and narratology, reviews the economics of the game industry, examines the aesthetics of game design, surveys the broad range of game genres, explores player culture, and addresses the major debates surrounding the medium, from educational benefits to the effects of violence.
\item[Web link:] \url{https://is.muni.cz/el/1421/podzim2016/IM082/Simon_Egenfeldt Nielson__Jonas_Heide_Smith__Susana_Pajares_Tosca_Understanding_Video_Games_The_Essential_Introduction_2008.pd}
\item[Full text link:]\url{https://is.muni.cz/el/1421/podzim2016/IM082/Simon_Egenfeldt Nielson__Jonas_Heide_Smith__Susana_Pajares_Tosca_Understanding_Video_Games_The_Essential_Introduction_2008.pdf}
\item[Comments:] 
\end{description}

\section*{Paper 4}
\begin{description}
\item[Title:] The Semiotics of Video Games
\item[Citation:] \cite{Bruchansky}
\item[Abstract:] What is the difference between a game and life? Is the game really ending when we go back to our everyday activities? Or could The Sims video game not be a good representation of our existence? It is with these questions in mind that I decided to explore the interdependence that exists between our everyday cultural reality and the rhetoric manifesting itself in video games. This paper introduces some of the key concepts used in the semiotics of video games and attempts to articulate them in a single frame. It is a short introduction to storyworlds, procedural rhetoric and gamespace. 
\item[Web link:]\url {https://www.researchgate.net/publication/315703967_The_Semiotics_of_Video_Games}
\item[Full text link:]\url {https://www.researchgate.net/publication/315703967_The_Semiotics_of_Video_Games}
\item[Comments:] 
\end{description}

\section*{Paper 5}
\begin{description}
\item[Title:] The Semiotic Immersion of Video Games, Gaming Technology and Interactive Strategies
\item[Citation:] \cite{nieva}
\item[Abstract:] The paper analyzes the effect of immersion in digital games using the theoretical apparatus of game theory. The paper illustrates interactive operations and the cause and effect relationship between player and designer, explaining the importance of strategic decision-making and pathing in player immersion. It considers the game function of creating a virtual world and proposes the idea that digital games are not just computer-mediated communication to the player. These games are games of “the moment”, like the game Chicken, and played with apparently great emotion, intelligence, and physical dexterity, although represented in software form. The relationship between the player and the computer is one of sign exchange, precisely the one that semiotics calls semiosis. The paper concludes that the personal achievement of individual players (end-users) accounts for the phenomenon of deep immersion in digital games. Not virtuality, but virtuosity is the strong force in digital game playing.
\item[Web link:]\url {http://pjos.org/index.php/pjos/article/view/8819}
\item[Full text link:]\url {http://citeseerx.ist.psu.edu/viewdoc/download?doi=10.1.1.869.8632&rep=rep1&type=pdf}
\item[Comments:] 
\end{description}

\section*{Paper 6}
\begin{description}
\item[Title:] Bungie fixing Destiny 2 armor resembling white nationalist symbol
\item[Citation:] \cite{samit}
\item[Abstract:]
\item[Web link:]\url {https://www.polygon.com/2017/9/12/16296256/destiny-2-armor-hate-group-symbol-kek-kekistan-flag}
\item[Full text link:]\url {https://www.polygon.com/2017/9/12/16296256/destiny-2-armor-hate-group-symbol-kek-kekistan-flag}
\item[Comments:] 
\end{description}

\section*{Paper 7}
\begin{description}
\item[Title:] Brown v. Entertainment Merchants Association
\item[Citation:] \cite{brownv}
\item[Abstract:]
\item[Web link:]\url {https://en.wikipedia.org/wiki/Brown_v._Entertainment_Merchants_Ass%27n}
\item[Full text link:] \url {https://en.wikipedia.org/wiki/Brown_v._Entertainment_Merchants_Ass%27n}
\item[Comments:]
\end{description}

\section*{Paper 8}
\begin{description}
\item[Title:] Four Ways to Use Symbols to Add Emotional Depth to Games
\item[Citation:] \cite{Depth}
\item[Abstract:]
\item[Web link:] \url {https://www.gamasutra.com/view/feature/131383/four_ways_to_use_symbols_to_add_.php}
\item[Full text link:]\url {https://www.gamasutra.com/view/feature/131383/four_ways_to_use_symbols_to_add_.php}
\item[Comments:] 
\end{description}

\section*{Paper 9}
\begin{description}
\item[Title:] When I Soar To Worlds Unknown: Symbolism and imagery in Myst
\item[Citation:] \cite{xander}
\item[Abstract:]
\item[Web link:]\url {https://www.gamasutra.com/blogs/XanderMarkham/20100515/87335/When_I_Soar_To_Worlds_Unknown_Symbolism_and_imagery_in_Myst.php}
\item[Full text link:]\url {https://www.gamasutra.com/blogs/XanderMarkham/20100515/87335/When_I_Soar_To_Worlds_Unknown_Symbolism_and_imagery_in_Myst.php}
\item[Comments:] 
\end{description}

\section*{Paper 10}
\begin{description}
\item[Title:] Bungie explains how a hate symbol ended up in Destiny 2
\item[Citation:] \cite{kerr}
\item[Abstract:]
\item[Web link:]\url  {https://www.gamasutra.com/view/news/305791/Bungie_explains_how_a_hate_symbol_ended_up_in_Destiny_2.php}
\item[Full text link:] \url {https://www.gamasutra.com/view/news/305791/Bungie_explains_how_a_hate_symbol_ended_up_in_Destiny_2.php}
\item[Comments:]
\end{description}

\section*{Paper 11}
\begin{description}
\item[Title:] Who's Afraid of the Swastika? Nazi Symbols in Video Games
\item[Citation:] \cite{jakub}
\item[Abstract:]
\item[Web link:] \url {https://www.gamepressure.com/e.asp?ID=57}
\item[Full text link:]\url {https://www.gamepressure.com/e.asp?ID=57}
\item[Comments:] 
\end{description}

\section*{Paper 12}
\begin{description}
\item[Title:] The Symbol Grounding Problem
\item[Citation:] \cite{harnad}
\item[Abstract:]
\item[Web link:]\url {http://www.summer10.isc.uqam.ca/page/docs/readings/HARNAD_Stevan/symgro.pdf}
\item[Full text link:]\url {http://www.summer10.isc.uqam.ca/page/docs/readings/HARNAD_Stevan/symgro.pdf}
\item[Comments:]
\end{description}

\bibliographystyle{ieeetran}
\bibliography{initial_references}

\end{document}
