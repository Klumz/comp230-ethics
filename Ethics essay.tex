% Please do not change the document class
\documentclass{scrartcl}

% Please do not change these packages
\usepackage[hidelinks]{hyperref}
\usepackage[none]{hyphenat}
\usepackage{setspace}
\doublespace

% You may add additional packages here
\usepackage{amsmath}

% Please include a clear, concise, and descriptive title
\title{When is it morally justified to retrospectively replace offensive symbolism in mainstream AAA games?}

% Please do not change the subtitle
\subtitle{COMP230 - Ethics Essay}

% Please put your student number in the author field
\author{1607934}

\begin{document}

\maketitle

\abstract{} 
While the game industry grows in numbers, so does its influence. There have been multiple cases of controversial symbolism in video games in the past (e.g. Wolfenstein 3D), however has become more prominent during these recent years. A brief example of this would be from the first-person shooter game Destiny 2, where a pair of gauntlets in-game were patched for allegedly resembling a hate symbol from a satiric "religion" known as Kek. Are such actions necessary? Do they contribute to the future betterment of the video games industry and its wide audience?

\section{Introduction}
Symbolism has been a huge part of video games throughout the years. Regardless of their positive or negative outcomes, they have influenced both the game industry and its audience. Symbolism can be used to enhance a player's experience within a game regarding the world, story and characters. It can be the main focus of the game or left as a more subtle way of enforcing the base foundation of its world. However, when it becomes too relatable and forms akin to real and controversial world events, should it be altered to avoid critical backlash and potential misinterpretation? Additionally, when would it be morally justified to replace such offensive and controversial symbolism, particularly in mainstream AAA games? This paper will focus on these topics.


\section{Symbolism}
Symbolism in its base form is a visual representation, e.g. a heart represents love, colours such as red representing danger and so on. It is used to provide depth by conveying it through such imagery. Video games in particular use and are dependant on these to reinforce characterisation and plot, adding extra weight to the story. A few examples of this would be from Myst, a game that contained a considerable amount of symbols that evoked our inner exploratory instincts as a child \cite{xander}. Another example involves the infamous cake from Portal, which was initially supposed to represent victory, although later turned out to be a lie by the game's villain, GLaDOS. Symbolism in video games is no different to other mediums such as books and movies, however, unlike in books and movies, the player has a personal role in the story that they themselves are able to alter. As a result, this makes everything more personal and therefore has a bigger impact. This is why it is important to expand one's own symbolic thinking \cite{xander}

\section{Freedom of speech}



\section{Understanding the intent}



\section{Conclusion}



\bibliographystyle{ieeetran}
\bibliography{references}

\end{document}
