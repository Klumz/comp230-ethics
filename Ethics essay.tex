% Please do not change the document class
\documentclass{scrartcl}

% Please do not change these packages
\usepackage[hidelinks]{hyperref}
\usepackage[none]{hyphenat}
\usepackage{setspace}
\doublespace

% You may add additional packages here
\usepackage{amsmath}

% Please include a clear, concise, and descriptive title
\title{When is it morally justified to retrospectively replace offensive symbolism in mainstream AAA games?}

% Please do not change the subtitle
\subtitle{COMP230 - Ethics Essay}

% Please put your student number in the author field
\author{1607934}

\begin{document}

\maketitle

\abstract{} 
While the game industry grows in numbers, so does its influence. There have been multiple cases of controversial symbolism in video games in the past (e.g. Wolfenstein 3D), however has become more prominent during these recent years \cite{jakub}. A brief example of this would be from the first-person shooter game Destiny 2, where a pair of gauntlets in-game were patched for allegedly resembling a hate symbol from a satiric "religion" known as Kek \cite{samit}. Are such actions necessary? Do they contribute to the future betterment of the video games industry and its wide audience?

\section{Introduction}
Video games Symbolism has been a huge part of video games throughout the years. Regardless of their positive or negative outcomes, they have influenced both the game industry and its audience. Symbolism can be used to enhance a player's experience within a game regarding the world, story and characters. It can be the main focus of the game or left as a more subtle way of enforcing the base foundation of its world. However, when it becomes too relatable and forms akin to real and controversial world events, should it be altered to avoid critical backlash and potential misinterpretation? Additionally, when would it be morally justified to replace such offensive and controversial symbolism, particularly in mainstream AAA games? This paper will focus on these topics.


\section{Symbolism}
Symbolism in its base form is a visual representation, e.g. a heart represents love, colours such as red representing danger and so on. It is used to provide depth by conveying it through such imagery, and is often done through storyworlds and interactable algorithms \cite{Bruchansky} \cite{nieva}. Video games in particular use and are dependant on these to reinforce characterisation and plot, adding extra weight to the story. A few examples of this would be from Myst, a game that contained a considerable amount of symbols that evoked our inner exploratory instincts as a child \cite{xander}. Another example involves the infamous cake from Portal, which was initially supposed to represent victory, although later turned out to be a lie by the game's villain, GLaDOS. Symbolism in video games is no different to other mediums such as books and movies, however, unlike in books and movies, the player has a personal role in the story they themselves are able to fill and alter. As a result, this makes everything more personal and therefore has a bigger impact. This is why it is important to further one's own symbolic thinking \cite{jefforkin}.

\section{Freedom of speech}
Previously, a 2011 case in California, US emerged in an effort to ban the sale of violent video games to minors \cite{brownv}. That was until they were ruled to be considered as art by law, being recognised as an element and part of our culture - bearing no difference, again, to other forms of media such as books and movies that were already protected beforehand. Further expanding and improving such laws internationally for symbolism in games is a notable consideration, as they promote free speech in the digital age and provide video game developers more freedom to fully express their ideas the way they see fit. Essentially, it is important to remember that video games are simply a form of art and expressions of various ideas brought together by single or multiple individuals, emphasising storytelling and interaction.


\section{Understanding intent}
Video games contain elaborate characters, stories and worlds drawn from either fiction or history. Players inhabit their virtual characters along with their objectives. In this mindset, on the basis of what you learned about the game's story and world, one must adapt to the world's and character's mental states, e.g. emotions, beliefs, values, etc. However, if the game's world is too alike that of reality, which many AAA games currently are with their vast open worlds and real-world portayals, it can intensify the player's perception of it as a result, giving room for personal criticism despite the game being a fictional world. It is therefore important to remember to detach from reality and personal beliefs in game worlds. A related issue within representationalism, brought up by Stevan Harnad, arises from the term \textit{symbol grounding} problem where the meanings of words and symbols that are connected to their referents are discussed and questioned, e.g. why does a red X mean "stop" or "no"? \cite{harnad} \cite{davidmyers}.


\section{Conclusion}
In conclusion, video games contain unique aspects, such as symbolism, that are important for conveying visual information in order for the player to understand their surroundings and circumstances. Games, especially AAA games, are not fully free from receiving backlash in their use of symbolism unless the players themselves understand the intentions behind them.



\bibliographystyle{ieeetran}
\bibliography{references}

\end{document}
